\chapter{Introduction}

\section{Motivation}

The motivation behind this project stems from my pursuit of an integrated Ph.D.
program, where I will focus on investigating the potential enhancements in detection capabilities for Autonomous Surface Vessels (ASVs) through the integration of data from various sensors, such as lidars and specialized cameras.
This research will heavily involve machine learning techniques, necessitating a significant amount of training data.
Since I will be exploring new sensor combinations, existing datasets alone cannot fulfill the requirements.
Hence, there is a need for a framework that enables the collection of new data, with a preference for a low threshold approach.
Therefore, the development of this sensor rig becomes imperative.


\section{TODO}
Explain how this project is a bit all over the place but show how it ties together with a full page figure.

\section{Challenges}
The main challenge in this project has been to design a system that can effectively manage the large amount of raw data produced by the two cameras.
Each camera features a 5.0MP sensor configured to capture 10-bit raw images 16 times every second \cite{lucidvisionlabsTriton0MPPolarization}.
To put this into perspective, a single frame contains more data than what is needed to store the entire collection of William Shakespeare's works in plain text \cite{projectgutenbergCompleteWorksWilliam1994}.

If the output from these cameras were stored directly to disk, almost a Terabyte of data would be generated every hour, making it infeasible to collect long term data sets.


\section{Outline}
\todo


\paragraph{Chapter \ref{chap:programming_theory}: Real time programming on Jetson Xavier}
This chapter provides an overview of the theoretical background necessary to achieve real-time performance on the \jx platform.
It covers topics such as real-time programming theory, performance considerations for heterogeneous systems, and a brief introduction to the CUDA programming model.

\paragraph{Chapter \ref{chap:cameras}: Polarization Cameras}
This chapter explains how interaction with the camera is acheived, how they are configured and how various the network adapters have been configured to acheive $2Gb/s$ throughput from the \cams.

\paragraph{Chapter \ref{chap:debayer}: Efficient processing of raw image bitstream in CUDA}
To make it possible to compress the video streams, the raw data has to be transformed into a format that can be processed by the \gls{h265} encoder.
This chapter explains how this is acheived in real time using CUDA.

\paragraph{Chapter \ref{chap:gstreamer}: Video Compression using GStreamer}
\gls{gstreamer} is a powerful framework for processing video streams.
This chapter explains how it is used to compress the video streams in real time using hardware accelration on the \jx.

\paragraph{Chapter \ref{chap:pipeline}: Pipeline Assembly in Python}
\todo


\paragraph{Chapter \ref{chap:gui}: Web Interface to Control, Monitor and Stream Video}
A web application was developed to enabling real time control and monitoring of the \sr and its video streams.
This allows anyone with a smartphone to operate the \sr without any prior training or technical knowledge.


\paragraph{Chapter \ref{chap:flashing_xavier}: Compiling Jetson Linux with PPS}
In order to get the necessary functionality out of the \jx, it was necessary to update the operating system to a newer version.
As a \gls{pps} signal is used to synchronize the clock on the \jx to \gls{utc}, it was necessary to compile the kernel with support for this, which turned out to be a non-trivial task.

\paragraph{Chapter \ref{chap:hardware}: Improving the Hardware}
After the \preproject most of the hardware was already in place, but some parts were missing.
I applied for funding to purchase a new 3D printer, which was then used to manufacture a custom ergonomic carry handles and new combined camera and antenna mounts for the \sr.

\paragraph{Chapter \ref{chap:results}: Results}


\paragraph{Chapter \ref{chap:future_work}: Ongoing developments and future work}



\section{Contributions}
Automated pipeline for compiling and flashing Jetson products.
High performance custom debayer algorithm for Polarization RGS cameras written in CUDA.
Development and manufacturing of ergonomic carry handles.
Software pipeling for real time image debayering and compression built on GStreamer.
Framework for creating real time graphical web interfaces.

\section{Proof of }
