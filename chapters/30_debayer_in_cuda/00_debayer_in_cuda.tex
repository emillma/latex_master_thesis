\chapter{Efficient Preprocessing of Raw Image Data in CUDA}
\label{chap:debayer}

To compress the video frames from the cameras, it is necessary to first transform the data into a format that is compatible with the encoder.
This involves debayering the data to extract color information, converting the extracted color into a different color space, and packaging the data in a specific format.
Since the cameras utilize a specialized image sensor, a custom preprocessor was developed to perform these operations.
The preprocessor takes the raw bytes from the cameras as input and generates bytes that can be directly sent to the H.265 encoder.
In order to achieve the required throughput and limit power consumption, the preprocessor is implemented using\gls{cuda}.

This chapter provides an overview of the theory behind debayering and color spaces and presents the optimized\gls{cuda} implementation.

