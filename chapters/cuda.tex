

\cuda is a parallel computing platform and programming model created by NVIDIA that allows developers to harness the power of GPU accelerators for various applications, from \gls{hpc} to \gls{ai} [Source 4](https://blogs.nvidia.com/blog/2012/09/10/what-is-cuda-2/).
It is an extension of C/C++ programming and provides an API that enables performing parallel computations on NVIDIA GPUs [Source 2](https://www.geeksforgeeks.org/introduction-to-cuda-programming/).
CUDA brings together massively parallel hardware, a C-based programming language, an assembly language, and a software development kit (SDK) that includes libraries, debugging, profiling, and compiling tools [Source 1](https://stackoverflow.com/questions/5211746/what-is-cuda-like-what-is-it-for-what-are-the-benefits-and-how-to-start).

The main advantage of CUDA is its ability to achieve significant performance improvements (up to 50x or more) for parallelizable tasks over traditional CPU processing [Source 1](https://stackoverflow.com/questions/5211746/what-is-cuda-like-what-is-it-for-what-are-the-benefits-and-how-to-start).
It provides a general-purpose language that makes it easier for non-GPU programmers to adopt, and it demonstrates NVIDIA's commitment to supporting general-purpose parallelization on their hardware [Source 1](https://stackoverflow.com/questions/5211746/what-is-cuda-like-what-is-it-for-what-are-the-benefits-and-how-to-start).
CUDA also offers integrated memory, shared memory, and improved performance for data transfers between the CPU and GPU [Source 2](https://www.geeksforgeeks.org/introduction-to-cuda-programming/).
However, it is essential to note that not all algorithms or programs can benefit from CUDA, as it is best suited for tasks that require simple, repetitive operations on massive numbers of threads or data points [Source 1](https://stackoverflow.com/questions/5211746/what-is-cuda-like-what-is-it-for-what-are-the-benefits-and-how-to-start).