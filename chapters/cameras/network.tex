\section{Network configuration}
The network configuration is important because each camera produces a lot of data.
The different network parameters are described in the following sections.



\subsection{Jumbo Frames}
An Ethernet frame is a unit of packetized formatted information that includes the Ethernet header, payload, and CRC trailer. \cite{winterCisco3ComApplied2009}.
The original Ethernet specification, IEEE 802.3 \cite{ieeeIEEEStandardsInterpretation2002}, allowed for a frame size between 64 to 1518 bytes, with a standard header length of 18 bytes.
Ethernet Jumbo frames carry more payload than the maximum specified by IEEE 802.3, with an \gls{mtu} size of up to 9000 bytes \cite{lucidvisionlabsJumboFramesLUCID2020}.
Increasing the maximal \gls{mtu} size typically leads to improved performance for high-bandwidth cameras and can also help reduce the CPU load on the host system \cite{lucidvisionlabsJumboFramesLUCID2020}, as there is less protocol overhead \cite{lukeThingsYouShould2018}.
Both the ethernet card used on the \jx as well as the \cams support Jumbo frames \cite{IntelI350am4Chipa} \cite{TritonMPPolarized2020a}.

\subsection{Receive Buffers}
\begin{figure}
    \centering
    \includegraphics[width=0.8\textwidth]{figures/linux_networking.png}
    \caption{Simplified high level overview of the queues on the transmit path of the Linux network stack \cite{danQueueingLinuxNetwork2013}}
    \label{fig:linux_network}
\end{figure}
When network packets arrive on Linux they are stored in a driver queue as shown in Figure \ref{fig:linux_network} \cite{danQueueingLinuxNetwork2013}.
To avoid starvation and increase performance, it is recommended to increase the default number of \glspl{skb}, also known as receive buffers \cite{lucidvisionlabsReceiveBuffers2020} \cite{danQueueingLinuxNetwork2013}.
This can be done using the \mintinline{bash}{ethtool -G} command \cite{danQueueingLinuxNetwork2013}.
The maximal allowed number of receive buffers on the \jx are 4096, opposed to the default value of 256, according to the \mintinline{bash}{ethtool -g eth1} command.
The poetntial downside of increasing the number of \glspl{skb} is that it increases the memory usage and it might introduce more Latency as the Driver Queue gets longer \cite{danQueueingLinuxNetwork2013}. As the current use case for the sensor platform is to record data, this is not a problem, but it should be kept in mind for future work if the sensor platform is used for real-time applications.

As we are receiving large amounts of data as jumbo frames it is also recommended to increase the default and the maximal socket buffer size.
The default values for the default and maximal receive buffer saize were \todo and \todo, which was found using the \mintinline{bash}{sysctl -a | grep rmem_default} and \mintinline{bash}{sysctl -a | grep rmem_max} commands on a newly booted \jx \cite{sainiUnableReadNet2021}.

\lucid suggests setting both default and max buffer size to $1MiB$ while IBM suggest setting them up to $16MB$ for best performance \cite{lucidvisionlabsReceiveBuffers2020}\cite{ibmIBMDocumentationTCPIP2021}.
As it is quite cumbersome to perform low level network latency analysis, and memory depletion never seemed to be an issue with the $16GB$ of \gls{ram} available on the \jx, both default and max buffer size was set to $16MiB$.
To make this change permanent the following command is used \cite{redhat10ChangingNetwork}.
\begin{minted}{bash}
    $ sudo sh -c "echo 'net.core.rmem_default=16777216' >> /etc/sysctl.conf"
    $ sudo sh -c "echo 'net.core.rmem_max=16777216' >> /etc/sysctl.conf"
    $ sudo sysctl -p
\end{minted}

\subsection{IP management}
The discovery and enumeration process of the \lucid cameras is outlined in Figure \ref{fig:lucid_ip_discovery}.

\subsubsection{Reverse Path Filtering}
\gls{rpf} is a security feature that is designed to prevent IP address spoofing attacks by discarding incoming network traffic that does not have a valid return path \cite{ReversePathFiltering}.
However, in some cases, such as when discovering cameras over \gls{gigev} using \gls{lla}, disabling \gls{rpf} may be necessary ensure detection \cite{lucidvisionlabsArenaSoftwareDevelopment2020}. This can for example occur if the cameras IP address collides with that of the network adapter \cite{lucidvisionlabsArenaSoftwareDevelopment2020}.

\subsection{Link-Local Adress}
In computer networking, \gls{lla} refers to an IP address that is used for communication within a logical division or broadcast domain to which the host is connected \cite{annieahujaweb2020LinkLocalAddress2022}.
\gls{lla} is unique within a network segment, but not across different network segments, and therefore should not be forwarded by routers \cite{annieahujaweb2020LinkLocalAddress2022}.

\gls{lla} is a method of IP address assignment that is commonly used when working with \gls{gigev} cameras \cite{teledyneSettingIPAddress01} \cite{lucidvisionlabsArenaSoftwareDevelopment2020}.
It allows devices to automatically assign IP addresses without the need for a \gls{dhcp} server \cite{annieahujaweb2020LinkLocalAddress2022}, which makes it possible to connect a \gls{gigev} camera directly to an ethernet adapter, without the need for an intermediate router.
For this to work properly the network adapter shoud be assigned the IP \code{169.254.0.1} and have netmask \code{255.255.0.0} as \gls{lla} addresses are assigned within the scope of \code{169.254.1.0} \code{169.254.254.255} \cite{annieahujaweb2020LinkLocalAddress2022}\cite{lucidvisionlabsArenaSoftwareDevelopment2020}.



\begin{figure*}
    \centering
    \includegraphics[height=\textheight]{figures/thing.pdf}
    \caption{Figure of the descovery and enumeration process of Lucid cameras \cite{TritonMPPolarized2020}}
    \label{fig:lucid_ip_discovery}
\end{figure*}