\section{Working with PyGobject}


\subsection{Type hints from typing stubs}
\gls{pygo} is a wrapper library \gs \gls{api} that lacks type hints by default.
This makes the development process slow as external documentation has to be consulted for every minor detail, like the name of a funcion or the order of its arguments.
On top of this, the syntax of the \gls{pygo} library is different from the syntax of the \gs \gls{api}.
The \gls{api} is written in C and uses a function oriented approach where pointers to objects are passed to functions, while \gls{pygo} is written in \py and is object oriented.
A illustrative example of this is how memory is appended to a buffer in the \gs \gls{api} and in \gls{pygo}, as shown in Listing \ref{lst:gst_memory_append_difference}.
\begin{listing}[H]
    \begin{minted}[linenos=false]{text}
        (C:)         gst_buffer_append_memory(buffer_ptr, memory_ptr)
        (Python:)    buffer.append_memory(memory)
    \end{minted}
    \caption{Difference between \gs \gls{api} and \gls{pygo} syntax}
    \label{lst:gst_memory_append_difference}
\end{listing}
To make the development process easier, the separate \code{pygobjcet-stubs} library, which provides type hints for \gls{pygo} should be used \cite{pygobjectTypingStubsPyGObject2023}.
