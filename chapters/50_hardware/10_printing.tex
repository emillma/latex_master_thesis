\section{3D printing}

\subsection{Motivation for buying a 3D printer}
The Department of Engineering Cybernetics is renting office space at Nyhavna to accomodate the growing number of employees.
Due to this relocation, I have to travel to Gløshaugen, which is a 40 minute walk away, if I want to use the workshops there.
During the preproject most of the parts were with the laser cutter at \gls{ov}, as is could be done in a single session.
The process of using a 3D printer was highly inconvenient as it necessitated two separate trips for each print job: one to initiate the print and another to stop it.
If the print failed, the process had to be repeated.
Based on this experience it was deemed necessary to have a 3D printer available at Nyhavna.
Argumentation was put forward to the department and I got funding to aquire a 3D printer at Nyhavna.

\subsection{Choice of printer}
Selecting a 3D printer is a difficult task as there are many different options available.
The main considerations were:
\begin{enumerate}
    \item \textbf{Availability:} The printer should be available for purchase.
    \item \textbf{Price:} The printer should be within the assigned budget.
    \item \textbf{Reliability:} The printer should be reliable and require little maintenance.
    \item \textbf{Print quality:} The printer should be able to print high quality parts.
    \item \textbf{Print volume:} The print volume shold be able to print relatively large parts.
    \item \textbf{Multi filament printing:} The printer should ideally be able to print with multiple filaments simultaniously.
\end{enumerate}



\subsection{Slicer}



\subsection{Parameters}
\subsubsection{Layer height}
\subsubsection{Infill}
\subsubsection{Initial layer height}




\section{Test/fix cooling}
Verify if cooling is sufficient.

\section{Camera mounts}

\subsection{3D printer}

\subsubsection{Workflow}
Having a 3D printer available changed the workflow compared to last year.
Last year everything was made at \gls{ov} as there was no real workshop available at

, so if parts were 3D printed two separate trips were requiered, one to start print and one to stop it.
Test parts were therefore not really an option.
This made using the laser cutter more favorable as multiple test parts could be made in one session.

\subsubsection{Snapmaker J1}
Idex printer makes it easier to print supports.
Bambu lab X1 and Prusa were considered.
Long delay.

\subsection{M109}


\subsection{Configuration}
Quite a lot of time was spent on configuring the printer.
Two different slicers were used, Cura and Luban.
Router was aquired for wireless transfer of files.

\subsection{Glass prind bed}
The \gls{j1} printer is delivered with a print


\subsection{Initial layer flow}
One issue was that material had a tendency to build up around the nozzle during the first layer as visualized in Figure \ref{fig:first_layer_buildup}.
This sometimes resulted in a dirty nozzle and some times the nozzle would pull the the material with it, resulting in a failed print.
Other people on the Snammaker forum have also reported this issue, without finding a clear solution \cite{artezioFinerAdjustmentsOffset2021} \cite{napsZHeightCalibrationOffset2023}.

After attempting the printer calibration process multiple times without resolving the issue, the next approach was to manually adjust the z-offset.
Unfortunately, the settings tab of the \gls{j1} did not provide any parameter for adjusting the z-offset.
The issue then soved by increasing the initial layer height in the slice profile and and reducing the initial layer flow until the issue dissapeared, which happened at a flow rate of $75\%$.
Subsequently, it was revealed that the z-offset adjustment on the \gls{j1} is indeed possible, albeit only accessible during the printing process for unknown reasons.
It was confirmed that the modifications made to the z-offset persisted between prints.



\begin{figure}[H]
    \centering
    \includegraphics[width=\textwidth]{figures/3d_print/first_layer_buildup.pdf}
    \caption{Illustration of material buildup on first layer.}
    \label{fig:first_layer_buildup}
\end{figure}


\subsection{Auto Towers}
\cite{kartchnerAutoTowersGeneratorUltimaker2022}